\documentclass[12pt, a4 paper]{report}
\usepackage[T1]{fontenc}
\usepackage[polish]{babel}
\usepackage[utf8]{inputenc}
\usepackage[T1]{fontenc}
\usepackage{titlesec}
\titleformat{\chapter}[display]
  {\normalfont\bfseries}{}{0pt}{\Huge}
\usepackage{blindtext}
\usepackage{xcolor}
\usepackage{float}
\usepackage{amsmath}
\usepackage{multirow}
\usepackage{graphicx}
\usepackage{caption}
\usepackage{subcaption}
\usepackage{hyperref}
\hypersetup{
    colorlinks=true,
    linkcolor=blue,
    pdftitle={Sprawozdanie I Kacper Piątkowski},
    pdfpagemode=FullScreen,
}
%\usepackage{subfigure}
\usepackage{gensymb}
\usepackage{amssymb}
\usepackage{xcolor}
\usepackage[margin=3.1cm]{geometry}
\titlespacing*{\chapter}{0pt}{-30mm}{40pt}
\usepackage{pgfplots}
\usepgfplotslibrary{units}
\pgfplotsset{width=14cm,compat=1.9}
\usepackage{fancyhdr}
\pagestyle{fancy}
\fancyhf{}
\lhead{Elektronika Cyfrowa}
\rhead{Kacper Piątkowski}
\fancyfoot[C]{\thepage}
\usepackage{lipsum}
\setcounter{chapter}{-1}

\title {Sprawozdanie I - poprawa\\
{\Large Elektronika Cyfrowa}}
\author{
Kacper Piątkowski\\
Grupa 4\\\\
Prowadzący: dr. Magdalena Skurzok}
\date{Data laboratoriów: 16.03.2022}
\label{grupa}
\begin{document}

\maketitle

\tableofcontents

\chapter{Poprawa}

\section{Informacje dotyczące PDFa}

\begin{itemize}
    \item Słowa zaznaczone na kolor \textcolor{blue}{niebieski} są hyperlinkami, najczęściej prowadzącymi do referencji używanych podczas sprawozdania.
    \item Hyperlinki znajdują się także w spisie treści.
    \item W poniżej umieszczonych informacjach dotyczących poprawy znajdują się hyperlinki do miejsc w których zostały dokonane zmiany.
\end{itemize}

\section{Informacje dotyczące poprawy}

Wszystkie poprawki wniesione na sprawozdanie zaznaczone są kolorem \textcolor{purple}{\textbf{purple}}. \\

Lista poprawek:
\begin{itemize}
    \item Dodanie numeru grupy ćwiczeniowej.
    \item Poprawienie odczytów z funkcji wbudowanych:
        \begin{itemize}
            \item 2.3 Pomiary: \ref{ad:odczyt_2_3}
        \end{itemize}
    \item Dodanie jawnego obliczania różnic pomiarowych
        \begin{itemize}
            \item 2.3 Pomiary: \ref{ad:roznica_2_3-1} \ref{ad:roznica_2_3-2} \ref{ad:roznica_2_3-3}
            \item 2.4 Zadanie praktyczne - pomiary: \ref{ad:roznica_2_4}
            \item 4.2 Dudnienia: \ref{ad:roznica_4_2-1}
        \end{itemize}
    \item Poprawa 'oscylator' na 'oscyloskop' \ref{ad:zla_nazwa_sprzetu_2_5}
    \item 3.1 Krzywe Lissajous - dodanie rysunków teoretycznych \ref{ad:teoretyczne_lissajous}
    \item 3.1 Krzywe Lissajous - dodanie informacji o parametrach sygnału \ref{ad:dodatkowe_informacje_lissajous}
\end{itemize}

\chapter{Wstęp}

\section{Używany sprzęt}
\begin{itemize}
    \item Stanowisko nr 7
    \item Oscyloskop - MSO3012 
    \item Generator funkcyjny - AFG3022B
    \item Płytka RLC - numer 15
\end{itemize}

\section {Jednostki i przedrostki}

\begin{itemize}
\item 1 \boldsymbol{\Omega} (om) - wielkość rezystancji - 1\boldsymbol{\Omega} = {$\frac{1V}{1A}$} ($\frac{wolt}{amper}$)
\item 1 V (wolt) - jednostka potencjału elektrycznego, napięcia elektrycznego i siły elektromotorycznej - 1V = $\frac{1W}{1A}$ ($\frac{wat}{amper}$)
\item 1 A (amper) - jednostka natężenia prądu elektrycznego - 1A = $\frac{1C}{1s}$
\item 1 Hz (herc) - jednostka miary częstotliwości - 1Hz = $\frac{1}{1s} = 1s^{-1}$
\end{itemize}

\begin{itemize}
    \item k (kilo) = $10^3$
    \item m (mili) = $10^{-3}$
\end{itemize}


\section {Oscyloskop}
Przyrząd elektroniczny służący do obserwowania, obrazowania i badania przebiegów
zależności pomiędzy dwiema wielkościami elektrycznymi, bądź innymi wielkościami fizycznymi
reprezentowanymi w postaci elektrycznej.
\section {Generator funkcyjny}

Urządzenie generujące oscylujące napięcie przy zadanych parametrach
częstotliwości, amplitudy oraz kształtu sygnału.

\section{Krzywe Lissajous}

Krzywa parametryczna, która jest wykreślana przez punkt materialny wykonujący
drgania harmoniczne w dwóch wzajemnie prostopadłych kierunkach dana wzorem:
\begin{equation}
    \begin{cases}
        x(t) = Asin(at + \delta) \\
        y(t) = Bsin(bt)
    \end{cases}
\end{equation}

\section {Multimetr (Miernik uniwersalny)}

Zespolone urządzenie pomiarowe posiadające możliwość pomiaru różnych wielkości fizycznych

\section {Zjawisko dudnień}

Zjawisko powstające w wyniku nałożenia się dwóch drgań o takich samych amplitudach i nieznacznie różniących się częstotliwościach.

\section {Łączenie rezystorów}

Szeregowe:
\begin{equation} \label{szereg}
    R_z = \sum^{n}_{i=1}{R_i}
\end{equation}
Równoległe:
\begin{equation}
    \frac{1}{R_z} = \sum^{n}_{i=1}{\frac{1}{{R_i}}}
\end{equation}

\section {Prawo Ohma}

Prawo fizyki głoszące proporcjonalność natężenia prądu płynącego przez przewodnik do napięcia panującego między końcami przewodnika.\\
Dla prądu stałego proporcjonalność napięcia U i natężenia I wyraża się wzorem:
\begin{equation}
    U = I * R
\end{equation}

\section {Dzielnik napięcia}

Obwód dostarczający na wyjściu napięcie będące ułamkiem napięcia wejsciowego. 

\chapter{Wzmacniacz operacyjny}


\section{Wstępne pomiary}
\begin{itemize}
    \item Na początku należało zapoznać się ze schematem ideowym układu wzmacniacza operacyjnego (\ref{fig:schemat_wzmacniacz_odwracający})
    \item Płytka ze wzmacniaczem:
        \begin{figure}[H]
            \centering
            \includegraphics[scale=0.25]{img/phone/1651502036878_scaled.png}
            \caption{}
            \label{fig:my_label}
        \end{figure}
    \item Wartości napięć na pinach +/-:
        \begin{gather}
            \label{pomiar:U+} U_+ = \textbf{11.96V} \\
            \label{pomiar:U-} U_- = \textbf{-12.02V}
        \end{gather}
    \end{itemize}

\section{Budowa układu}
\begin{itemize}
    \item Należało złożyć wzmacniacz o wzmocnieniu \textbf{K = 10}. Zgodnie ze wzorem \ref{wzor:napiecie_wyjsciowe_odwracajacy} należało dobrać odpowiednie oporniki, tak aby ich iloraz $\dfrac{R_2}{R_1}$ = 10.
    \item Dobrano dwa dodatkowe rezystory:
        \begin{gather}
            \label{wzmacniacz:R1} R_1 = \textbf{2.96k}\boldsymbol{\Omega} \\
            \label{wzmacniacz:R2} R_2 = \textbf{32.2k}\boldsymbol{\Omega}
        \end{gather}
        
    \pagebreak
        
    \item Układ złożono korzystając ze schematu (\ref{fig:schemat_wzmacniacz_odwracający}) oraz w/w oporników (\ref{wzmacniacz:R1}, \ref{wzmacniacz:R2}):
        \begin{figure}[H]
            \centering
            \begin{subfigure}[h]{0.4\textwidth}
                \includegraphics[width=\textwidth]{img/phone/1651502036867_scaled.png}
            \end{subfigure}
            \begin{subfigure}[h]{0.4\textwidth}
                \includegraphics[width=\textwidth]{img/phone/1651502036857_scaled.png}
            \end{subfigure}
            \caption{Zmontowany układ}
            \label{wzmacniacz:zmontowany_uklad}
        \end{figure}
\end{itemize}

\section{Testowanie poprawności układu oraz wartości wzmocnienia}
\begin{itemize}
    \item Za pomocą generatora funkcyjnego przesłano na płytkę falę sinusoidalną o parametrach: \textbf{1kHz} oraz \textbf{1V}. Ten sam sygnał za pomocą trójnika wyprowadzono do \textbf{kanału 1}, wyjściowy sygnał poprowadzono do \textbf{kanału 2}
    \item Odczytane za pomocą funkcji wbudowanych parametry sygnałów na oscyloskopie wyniosły:
        \begin{gather}
            \label{wzmacniacz_parametry:czestotliwosc}f_{we} = 1.003kHz = \textbf{1003Hz} \\
            \label{wzmacniacz_parametry:u_we} U_{we} = 952mV = \textbf{0.952V} \\
            \label{wzmacniacz_parametry:u_wy} U_{wy} = \textbf{10.6V} \\
            \label{wzmacniacz_parametry:przesuniecie}\phi_{12} = \textbf{-162.4}\boldsymbol{\degree}
        \end{gather}
        \begin{figure}[H]
            \centering
            \includegraphics[scale=0.4]{img/osciloscope/dzialajacy_wzmacniacz_cropped.png}
            \caption{}
            \label{fig:dzialajacy_wzmacniacz}
        \end{figure}
    \item Korzystając ze wzoru na teoretyczne napięcie wyjściowe (\ref{wzor:napiecie_wyjsciowe_odwracajacy}) oraz dobranych rezystorów (\ref{wzmacniacz:R1}, \ref{wzmacniacz:R2}):
        \begin{equation}
            K_{teor} = \dfrac{32.2k\Omega}{2.96k\Omega} = \textbf{10.87} 
        \end{equation}
    \item Korzystając z przeprowadzonych pomiarów (\ref{wzmacniacz_parametry:u_we}, \ref{wzmacniacz_parametry:u_wy}):
        \begin{equation}
            K_{eksp} = \dfrac{U_{wy}}{U_{we}} = \dfrac{10.6V}{0.952V} = \textbf{11.04}
        \end{equation}
    \item Sygnał po przejściu przez zbudowany układ (\ref{wzmacniacz:zmontowany_uklad}) zostaje poprawnie wzmocniony (\ref{fig:dzialajacy_wzmacniacz}).
    \item Zgodnie z założeniem sygnał wyjściowy jest przesunięty w fazie względem pierwszego o \textbf{162.4}$\boldsymbol{\degree}$
\end{itemize}

\section{Charakterystyka częstotliwościowa}

\begin{itemize}
    \item Eksperymentalnie wyznaczono wartość częstotliwości $\mathbf{f_g}$ dla której sygnał wyjściowy jest dużo \textbf{niższy} od wejściowego i wyniósł:
        \begin{center}
            \label{wzmacniacz:f_g} $\mathbf{f_g}$ = \textbf{1MHz}
        \end{center}
    Sygnał wyjściowy w tym przypadku wynosił $\approx 25\%$ sygnału wejściowego ($U_{we} = 1V$, $U_{wy} = 0.247V$)
    \item Następnie zmierzono wartości $\mathbf{U_{we}}$ $\mathbf{U_{wy}}$ $\mathbf{\boldsymbol{\phi}_{12}}$ w następujących częstotliwościach:
    \begin{center}
        \Large
        \begin{tabular}{|c|c|c|}
            \hline
            $f_p$[kHz] & $f_k$[kHz] & skok[kHz] \\
            \hline
            0 & 1 & 0.1 \\
            \hline
            1 & 10 & 1 \\
            \hline
            10 & 100 & 10 \\
            \hline
            100 & $f_g$ (\ref{wzmacniacz:f_g}) & 100 \\
            \hline
        \end{tabular}
    \end{center}
    
    \newpage
    \item Zmierzone wartości w postaci tabularycznej:
    \begin{center}
    %tabelka Uwe Uwy phi
    \begin{tabular}{|c|c|c|c|c|}
         \hline
         $f$ [kHz] & $U_{we}$ [V] & $U_{wy}$ [V] & $\frac{U_{wy}}{U_{we}}$ &  $\phi_{12}$ [$\degree$] \\
         \hline
         0.1 & 0.976 & 10.6 & 10.861 &  -64.67 \\
         \hline
         0.2 & 0.976 & 10.6 & 10.861 & -107.5 \\
         \hline
         0.3 & 0.976 & 10.6 & 10.861 & -136.9 \\
         \hline
         0.348 & 0.976 & 10.6 & 10.861 & -143.6 \\
         \hline
         0.444 & 0.976 & 10.6 & 10.861 & -133.5 \\
         \hline
         0.6 & 0.976 & 10.6 & 10.861 & -145 \\
         \hline
         0.764 & 0.976 & 10.6 & 10.861 & -145 \\
         \hline
         0.756 & 0.976 & 10.6 & 10.861 & -135 \\
         \hline
         0.780 & 0.976 & 10.6 & 10.861 & -134.7 \\
         \hline
         0.948 & 0.976 & 10.6 & 10.861 & -143.7 \\
         \hline
         2 & 0.976 & 10.6 & 10.861 & -178.3 \\
         \hline
         2.7 & 0.976 & 10.6 & 10.861 & -177.8 \\
         \hline
         4 & 0.976 & 10.6 & 10.861 & -176.4 \\
         \hline
         4.358 & 0.976 & 10.6 & 10.861 & -176.1 \\
         \hline
         5.33 & 0.976 & 10.6 & 10.861 & -175 \\
         \hline
         6.533 & 0.976 & 10.5 & 10.758 & -173.8 \\
         \hline
         7.993 & 0.976 & 10.5 & 10.758 & -172.2 \\
         \hline
         8.546 & 0.976 & 10.5 & 10.758 & -171.3 \\
         \hline
         9.431 & 0.976 & 10.4 & 10.656 & -170.2 \\
         \hline
         20.01 & 0.984 & 9.97 & 10.132 & -155.4 \\
         \hline
         24.35 & 0.991 & 9.11 & 9.193 & -147.4 \\
         \hline
         40 & 1 & 6.27 & 6.27 & -122.5 \\
         \hline
         49.98 & 1 & 5.03 & 5.030 & -114.9 \\
         \hline
         55.18 & 1 & 4.74 & 4.74 & -112.8 \\
         \hline
         70.08 & 1 & 3.78 & 3.78 & -107.1 \\
         \hline
         77.89 & 1 & 3.39 & 3.39 & -104.8 \\
         \hline
         82.42 & 1 & 3.23 & 3.23 & -103.5 \\
         \hline
         99.78 & 1 & 2.65 & 2.65 & -100.6 \\
         \hline
         200 & 1 & 1.3 & 1.3 & -92.03 \\
         \hline
         300.1 & 1 & 0.865 & 0.865 & -91.09 \\
         \hline
         399.7 & 1 & 0.632 & 0.632 & -89.14 \\
         \hline
         441.8 & 1 & 0.579 & 0.579 & -89.35 \\
         \hline
         542.4 & 1 & 0.405 & 0.405 & -90.19 \\
         \hline
         633.6 & 1 & 0.338 & 0.338 & -88.61 \\
         \hline
         800.2 & 1 & 0.290 & 0.29 & -56.30 \\
         \hline
         848.7 & 1 & 0.271 & 0.271 & -73.37 \\
         \hline
         926.6 & 1 & 0.247 & 0.247 & -88.71 \\
         \hline
    \end{tabular}
    \end{center}
    \pagebreak
    
    \item Niektóre z pomiarów:
        %osciloscope_figures 
        { 
        \begin{figure}[H]
            \centering
            \begin{subfigure}[h]{0.4\textwidth}
                \includegraphics[width=\textwidth]{img/osciloscope/charakterystyka/1_500_cropped.png}
                \caption*{500Hz}
            \end{subfigure}
            \begin{subfigure}[h]{0.4\textwidth}
                \includegraphics[width=\textwidth]{img/osciloscope/charakterystyka/1_1khz_cropped.png}
                \caption*{1kHz}
            \end{subfigure}
        \end{figure}
        
        \begin{figure}[H]
            \centering    
            \begin{subfigure}[h]{0.4\textwidth}
                \includegraphics[width=\textwidth]{img/osciloscope/charakterystyka/1_5khz_cropped.png}
                \caption*{5kHz}
            \end{subfigure}
            \begin{subfigure}[h]{0.4\textwidth}
                \includegraphics[width=\textwidth]{img/osciloscope/charakterystyka/1_10khz_cropped.png}
                \caption*{10kHz}
            \end{subfigure}
        \end{figure}
        
        \begin{figure}[H]
            \centering
            \begin{subfigure}[h]{0.4\textwidth}
                \includegraphics[width=\textwidth]{img/osciloscope/charakterystyka/1_50khz_cropped.png}
                \caption*{50kHz}
            \end{subfigure}
            \begin{subfigure}[h]{0.4\textwidth}
                \includegraphics[width=\textwidth]{img/osciloscope/charakterystyka/1_100khz_cropped.png}
                \caption*{100kHz}
            \end{subfigure}
        \end{figure}
        
        \begin{figure}[H]
            \centering
            \begin{subfigure}[h]{0.4\textwidth}
                \includegraphics[width=\textwidth]{img/osciloscope/charakterystyka/1_500khz_cropped.png}
                \caption*{500kHz}
            \end{subfigure}
            \begin{subfigure}[h]{0.4\textwidth}
                \includegraphics[width=\textwidth]{img/osciloscope/charakterystyka/1_1mhz_cropped.png}
                \caption*{1Mhz}
            \end{subfigure}
        \end{figure}
        }
    \pagebreak
    
    \item Charakterystyka amplitudowa (eksperymentalna oraz teoretyczna)
    \begin{figure}[H]
        \centering
        \begin{subfigure}[h]{0.45\textwidth}
            \includegraphics[width=\textwidth]{img/osciloscope/charakterystyka/uwy-we.png}
            \caption*{eksperymentalna}
        \end{subfigure}
        \begin{subfigure}[h]{0.45\textwidth}
            \includegraphics[width=\textwidth]{img/theoretical/charakterystyka_amplitudowa_wzmacniacz.png}
            \caption*{teoretyczna}
        \end{subfigure}
    \end{figure}
    
    \item Charakterystyka fazowa (eksperymentalna oraz teoretyczna)
        \begin{figure}[H]
        \centering
        \begin{subfigure}[h]{0.45\textwidth}
            \includegraphics[width=\textwidth]{img/osciloscope/charakterystyka/fazowa.png}
            \caption*{eksperymentalna}
        \end{subfigure}
        \begin{subfigure}[h]{0.45\textwidth}
            \includegraphics[width=\textwidth]{img/theoretical/charakterystyka_fazowa_wzmacniacz.png}
            \caption*{teoretyczna}
        \end{subfigure}
    \end{figure}
    \item Obie charakterystyki zgadzają się z przewidywaniami teoretycznymi, lecz widać na nich pewne odchylenia spowodowane niedokładnością pomiaru, która spowodowana może być złą kalibracją sprzętu, bądź za krótkim okresem kalibracji pomiędzy poszczególnymi pomiarami.
    \item Charakterystyka amplitudowa nie przyjmuje zamierzonego kształtu przez to, że pasmo częstotliwości jest większe.
\end{itemize}

\chapter{Badanie układów TTL}

\section{Układ 7400 (NAND)}

\begin{itemize}
    \item Schemat pinów dla układu \textbf{TTL 7400} (NAND):
        \begin{figure}[H]
            \centering
            \includegraphics[scale=0.25]{img/schemes/NAND_7400_pins.png}
            \caption{Piny 7400 (NAND)}
            \label{NAND:piny}
        \end{figure}
    \item Według schematu w układzie znajdują się: 
        \begin{center}
            pin uziemiający (\textbf{GND}) - pin \textbf{7} \\
            pin zasilający (\textbf{VCC}) - pin \textbf{14} \\
            piny wejścia (\textbf{iA}, \textbf{iB}) \\
            piny wyjścia (\textbf{iY}) \\
            \textbf{gdzie i - numer bramki logicznej}
        \end{center}
    \item Podpięto zasilanie oraz uziemienie. Podpinając wyjście z \textbf{1} impulsatora (wyższego) do \textbf{pinu 1}, a wyjście z \textbf{2} impulsatora do \textbf{pinu 2} oraz wyjście (\textbf{pin 3}) do próbnika stanów logicznych przetestowano bramkę NAND.
        \begin{figure}[H]
            \centering
                \begin{subfigure}[h]{0.4\textwidth}
                    \includegraphics[width=\textwidth]{img/NAND/test/1652306732892_scaled.png}
                \end{subfigure}
                \begin{subfigure}[h]{0.4\textwidth}
                    \includegraphics[width=\textwidth]{img/NAND/test/1652306732884_scaled.png}
                \end{subfigure}
                \begin{subfigure}[h]{0.4\textwidth}
                    \includegraphics[width=\textwidth]{img/NAND/test/1652306732872_scaled.png}
                \end{subfigure}
                \begin{subfigure}[h]{0.4\textwidth}
                    \includegraphics[width=\textwidth]{img/NAND/test/1652306732862_scaled.png}
                \end{subfigure}
        \end{figure}

\pagebreak

    \item Tabela stanów wyznaczona doświadczalnie:
        \begin{center}
            \begin{tabular}{|c|c|>{\columncolor[gray]{0.8}}c|}
                \hline
                A & B & Y \\
                \hline
                0 & 0 & 1 \\
                \hline
                0 & 1 & 1 \\
                \hline
                1 & 0 & 1 \\
                \hline
                1 & 1 & 0 \\
                \hline
            \end{tabular}
        \end{center}
    \item Napięcia wyjściowe dla logicznych 0 (L) oraz 1 (H):
        \begin{figure}[H]
            \centering
                \begin{subfigure}[h]{0.49\textwidth}
                    \includegraphics[width=\textwidth]{img/NAND/test/1652306732837_scaled.png}
                    \caption*{Pomiar dla L}
                \end{subfigure}
                \begin{subfigure}[h]{0.49\textwidth}
                    \includegraphics[width=\textwidth]{img/NAND/test/1652306732851_scaled.png}
                    \caption*{Pomiar dla H}
                \end{subfigure}
            \label{płytka:pomiar_impulsatorów}
        \end{figure}
        \begin{gather}
            U_{low} = \textbf{0.071V} \\
            U_{high} = \textbf{3.568V}
        \end{gather}
    \item Napięcia zgadzają się ze standardem TTL (\autoref{TTL:charakterystyka}).
\end{itemize}

\pagebreak

\section{Układ 7402 (NOR)}

\begin{itemize}
    \item Schemat pinów dla układu \textbf{TTL 7402} (NOR):
        \begin{figure}[H]
            \centering
            \includegraphics[scale=0.25]{img/schemes/NOR_7402_pins.png}
            \caption{Piny 7402 (NOR)}
            \label{NOR:piny}
        \end{figure}
    \item Według schematu w układzie znajdują się: 
        \begin{center}
            pin uziemiający (\textbf{GND}) - pin \textbf{7} \\
            pin zasilający (\textbf{VCC}) - pin \textbf{14} \\
            piny wejścia (\textbf{iA}, \textbf{iB}) \\
            piny wyjścia (\textbf{iY}) \\
            \textbf{gdzie i - numer bramki logicznej}
        \end{center}
    \item W odróżnieniu od schematu dla układu NAND (\ref{NAND:piny}) wyjścia z impulsatorów podpięto do pinów \textbf{2},\textbf{3}, zaś wyjście (\textbf{pin 1}) do próbnika stanów logicznych.
        \begin{figure}[H]
            \centering
                \begin{subfigure}[h]{0.4\textwidth}
                    \includegraphics[width=\textwidth]{img/NOR/test/1652306732823_scaled.png}
                \end{subfigure}
                \begin{subfigure}[h]{0.4\textwidth}
                    \includegraphics[width=\textwidth]{img/NOR/test/1652306732810_scaled.png}
                \end{subfigure}
                \begin{subfigure}[h]{0.4\textwidth}
                    \includegraphics[width=\textwidth]{img/NOR/test/1652306732798_scaled.png}
                \end{subfigure}
                \begin{subfigure}[h]{0.4\textwidth}
                    \includegraphics[width=\textwidth]{img/NOR/test/1652306732785_scaled.png}
                \end{subfigure}
        \end{figure}

\pagebreak

    \item Tabela stanów wyznaczona doświadczalnie:
        \begin{center}
            \begin{tabular}{|c|c|>{\columncolor[gray]{0.8}}c|}
                \hline
                A & B & Y \\
                \hline
                0 & 0 & 1 \\
                \hline
                0 & 1 & 0 \\
                \hline
                1 & 0 & 0 \\
                \hline
                1 & 1 & 0 \\
                \hline
            \end{tabular}
        \end{center}
    \item Napięcia wyjściowe dla logicznych 0 (L) oraz 1 (H):
        \begin{figure}[H]
            \centering
                \begin{subfigure}[h]{0.49\textwidth}
                    \includegraphics[width=\textwidth]{img/NOR/test/1652306732756_scaled.png}
                    \caption*{Pomiar dla L}
                \end{subfigure}
                \begin{subfigure}[h]{0.49\textwidth}
                    \includegraphics[width=\textwidth]{img/NOR/test/1652306732772_scaled.png}
                    \caption*{Pomiar dla H}
                \end{subfigure}
            \label{płytka:pomiar_impulsatorów}
        \end{figure}
        \begin{gather}
            U_{low} = \textbf{0.182V} \\
            U_{high} = \textbf{3.525V}
        \end{gather}
    \item Napięcia zgadzają się ze standardem TTL (\autoref{TTL:charakterystyka}).
\end{itemize}

\pagebreak

\section{Układ 7486 (XOR)}

\begin{itemize}
    \item Schemat pinów dla układu \textbf{TTL MC74HC86} (XOR):
        \begin{figure}[H]
            \centering
            \includegraphics[scale=0.25]{img/schemes/XOR_mc74hc86_pins.png}
            \caption{Piny 7486 (XOR)}
            \label{XOR:piny}
        \end{figure}
    \item Według schematu w układzie znajdują się: 
        \begin{center}
            pin uziemiający (\textbf{GND}) - pin \textbf{7} \\
            pin zasilający (\textbf{VCC}) - pin \textbf{14} \\
            piny wejścia (\textbf{iA}, \textbf{iB}) \\
            piny wyjścia (\textbf{iY}) \\
            \textbf{gdzie i - numer bramki logicznej}
        \end{center}
    \item Piny wejściowe oraz wyjściowe ułożone \textbf{są tak samo jak} dla schematu dla układu \textbf{NAND} (\ref{NAND:piny}) wyjścia z impulsatorów podpięto do pinów 1,2, zaś wyjście (pin 3) do próbnika stanów logicznych.
        \begin{figure}[H]
            \centering
                \begin{subfigure}[h]{0.4\textwidth}
                    \includegraphics[width=\textwidth]{img/XOR/test/1652306732731_scaled.png}
                \end{subfigure}
                \begin{subfigure}[h]{0.4\textwidth}
                    \includegraphics[width=\textwidth]{img/XOR/test/1652306732718_scaled.png}
                \end{subfigure}
                \begin{subfigure}[h]{0.4\textwidth}
                    \includegraphics[width=\textwidth]{img/XOR/test/1652306732707_scaled.png}
                \end{subfigure}
                \begin{subfigure}[h]{0.4\textwidth}
                    \includegraphics[width=\textwidth]{img/XOR/test/1652306732690_scaled.png}
                \end{subfigure}
        \end{figure}

\pagebreak

    \item Tabela stanów wyznaczona doświadczalnie:
        \begin{center}
            \begin{tabular}{|c|c|>{\columncolor[gray]{0.8}}c|}
                \hline
                A & B & Y \\
                \hline
                0 & 0 & 1 \\
                \hline
                0 & 1 & 0 \\
                \hline
                1 & 0 & 0 \\
                \hline
                1 & 1 & 0 \\
                \hline
            \end{tabular}
        \end{center}
    \item Napięcia wyjściowe dla logicznych 0 (L) oraz 1 (H):
        \begin{figure}[H]
            \centering
                \begin{subfigure}[h]{0.4\textwidth}
                    \includegraphics[width=\textwidth]{img/XOR/test/1652306732674_scaled.png}
                    \caption*{Pomiar dla L}
                \end{subfigure}
                \begin{subfigure}[h]{0.4\textwidth}
                    \includegraphics[width=\textwidth]{img/XOR/test/1652306732649_scaled.png}
                    \caption*{Pomiar dla H}
                \end{subfigure}
            \label{płytka:pomiar_impulsatorów}
        \end{figure}
        \begin{gather}
            U_{low} = \textbf{0.057V} \\
            U_{high} = \textbf{4.98V}
        \end{gather}
    \item Napięcia zgadzają się ze standardem TTL (\autoref{TTL:charakterystyka}).
\end{itemize}

\chapter{Budowanie funkcji logicznych (NAND)}

    \section{NOT}
        \begin{itemize}
            \item Zbudowany układ wraz ze schematem oraz rozpiską pinów:
                \begin{figure}[H]
                    \centering
                    \includegraphics[width=0.5\textwidth]{img/schemes_with_pins/NAND_not_w_pins.png}
                    \label{NAND:schemat_not_w_pins}
                \end{figure}
                \begin{figure}[H]
                    \centering
                    \includegraphics[width=0.5\textwidth]{img/NAND/funkcje/1652306732632_scaled.png}
                    \caption{Zbudowany układ}
                    \label{NAND:zbudowany_układ_NOT}
                \end{figure}
            \item Sprawdzanie stanów:
                \begin{figure}[H]
                    \centering
                        \begin{subfigure}[h]{0.35\textwidth}
                            \includegraphics[width=\textwidth]{img/NAND/funkcje/1652306732632_scaled.png}
                        \end{subfigure}
                        \begin{subfigure}[h]{0.35\textwidth}
                            \includegraphics[width=\textwidth]{img/NAND/funkcje/1652306732620_scaled.png}
                        \end{subfigure}
                    \label{NAND:testy_NOT}
                \end{figure}
            \item Tabela stanów:
                \begin{center}
                    \label{NAND:tabela_prawdy_NOT}
                    \begin{tabular}{|c|>{\columncolor[gray]{0.8}}c|}
                        \hline
                        A & Y \\
                        \hline
                        0 & 1 \\
                        \hline
                        1 & 0 \\
                        \hline
                    \end{tabular}
                \end{center}
            \item Zbudowany układ \textbf{poprawnie} reaguje na przesyłane sygnały (zgodnie z teoretycznymi przewidywaniami \ref{tabela_prawdy:NOT})
        \end{itemize}
    
\pagebreak

    \section{OR}
        \begin{itemize}
            \item Zbudowany układ wraz ze schematem oraz rozpiską pinów:
                \begin{figure}[H]
                    \centering
                    \includegraphics[width=\textwidth]{img/schemes_with_pins/NAND_or_w_pins.png}
                    \label{NAND:schemat_or_w_pins}
                \end{figure}
                \begin{figure}[H]
                    \centering
                    \includegraphics[width=\textwidth]{img/NAND/funkcje/1652306732527_scaled.png}
                    \caption{Zbudowany układ}
                    \label{NAND:zbudowany_układ_OR}
                \end{figure}
                        
            \pagebreak
            
            \item Sprawdzanie stanów:
                \begin{figure}[H]
                    \centering
                        \begin{subfigure}[h]{0.4\textwidth}
                            \includegraphics[width=\textwidth]{img/NAND/funkcje/1652306732527_scaled.png}
                        \end{subfigure}
                        \begin{subfigure}[h]{0.4\textwidth}
                            \includegraphics[width=\textwidth]{img/NAND/funkcje/1652306732517_scaled.png}
                        \end{subfigure}
                        \begin{subfigure}[h]{0.4\textwidth}
                            \includegraphics[width=\textwidth]{img/NAND/funkcje/1652306732506_scaled.png}
                        \end{subfigure}
                        \begin{subfigure}[h]{0.4\textwidth}
                            \includegraphics[width=\textwidth]{img/NAND/funkcje/1652306732496_scaled.png}
                        \end{subfigure}
                    \label{NAND:testy_OR}
                \end{figure}
            \item Tabela stanów:
                \begin{center}
                    \label{NAND:tabela_prawdy_OR}
                    \begin{tabular}{|c|c|>{\columncolor[gray]{0.8}}c|}
                        \hline
                        A & B & Y \\
                        \hline
                        0 & 0 & 0 \\
                        \hline
                        0 & 1 & 1 \\
                        \hline
                        1 & 0 & 1 \\
                        \hline
                        1 & 1 & 1 \\
                        \hline
                    \end{tabular}
                \end{center}
            \item Zbudowany układ \textbf{poprawnie} reaguje na przesyłane sygnały (zgodnie z teoretycznymi przewidywaniami \ref{tabela_prawdy:OR})
        \end{itemize}
        
\pagebreak

    \section{AND}
        \begin{itemize}
            \item Zbudowany układ wraz ze schematem oraz rozpiską pinów:
                \begin{figure}[H]
                    \centering
                    \includegraphics[width=\textwidth]{img/schemes_with_pins/NAND_and_w_pins.png}
                    \label{NAND:schemat_and_w_pins}
                \end{figure}
                \begin{figure}[H]
                    \centering
                    \includegraphics[width=\textwidth]{img/NAND/funkcje/1652306732610_scaled.png}
                    \caption{Zbudowany układ}
                    \label{NAND:zbudowany_układ_AND}
                \end{figure}
                
            \pagebreak
                
            \item Sprawdzanie stanów:
                \begin{figure}[H]
                    \centering
                        \begin{subfigure}[h]{0.4\textwidth}
                            \includegraphics[width=\textwidth]{img/NAND/funkcje/1652306732595_scaled.png}
                        \end{subfigure}
                        \begin{subfigure}[h]{0.4\textwidth}
                            \includegraphics[width=\textwidth]{img/NAND/funkcje/1652306732585_scaled.png}
                        \end{subfigure}
                        \begin{subfigure}[h]{0.4\textwidth}
                            \includegraphics[width=\textwidth]{img/NAND/funkcje/1652306732564_scaled.png}
                        \end{subfigure}
                        \begin{subfigure}[h]{0.4\textwidth}
                            \includegraphics[width=\textwidth]{img/NAND/funkcje/1652306732552_scaled.png}
                        \end{subfigure}
                    \label{NAND:testy_AND}
                \end{figure}
            \item Tabela stanów:
                \begin{center}
                    \label{NAND:tabela_prawdy_AND}
                    \begin{tabular}{|c|c|>{\columncolor[gray]{0.8}}c|}
                        \hline
                        A & B & Y \\
                        \hline
                        0 & 0 & 0 \\
                        \hline
                        0 & 1 & 0 \\
                        \hline
                        1 & 0 & 0 \\
                        \hline
                        1 & 1 & 1 \\
                        \hline
                    \end{tabular}
                \end{center}
            \item Zbudowany układ \textbf{poprawnie} reaguje na przesyłane sygnały (zgodnie z teoretycznymi przewidywaniami \ref{tabela_prawdy:AND})
    \end{itemize}

\pagebreak

\chapter{Budowanie funkcji logicznych (NOR)}

    \section{NOT}
        \begin{itemize}
            \item Zbudowany układ wraz ze schematem oraz rozpiską pinów:
                \begin{figure}[H]
                    \centering
                    \includegraphics[width=0.5\textwidth]{img/schemes_with_pins/NOR_not_w_pins.png}
                    \label{NOR:schemat_not_w_pins}
                \end{figure}
                \begin{figure}[H]
                    \centering
                    \includegraphics[width=0.5\textwidth]{img/NOR/funkcje/1652306732483_scaled.png}
                    \caption{Zbudowany układ}
                    \label{NOR:zbudowany_układ_NOT}
                \end{figure}
            \item Sprawdzanie stanów:
                \begin{figure}[H]
                    \centering
                        \begin{subfigure}[h]{0.4\textwidth}
                            \includegraphics[width=\textwidth]{img/NOR/funkcje/1652306732483_scaled.png}
                        \end{subfigure}
                        \begin{subfigure}[h]{0.4\textwidth}
                            \includegraphics[width=\textwidth]{img/NOR/funkcje/1652306732471_scaled.png}
                        \end{subfigure}
                    \label{NOR:testy_NOT}
                \end{figure}
            \item Tabela stanów:
                \begin{center}
                    \label{NOR:tabela_prawdy_NOT}
                    \begin{tabular}{|c|>{\columncolor[gray]{0.8}}c|}
                        \hline
                        A & Y \\
                        \hline
                        0 & 1 \\
                        \hline
                        1 & 0 \\
                        \hline
                    \end{tabular}
                \end{center}
            \item Zbudowany układ \textbf{poprawnie} reaguje na przesyłane sygnały (zgodnie z teoretycznymi przewidywaniami \ref{tabela_prawdy:NOT})
        \end{itemize}


\pagebreak
    
    \section{OR}
        \begin{itemize}
            \item Zbudowany układ wraz ze schematem oraz rozpiską pinów:
                \begin{figure}[H]
                    \centering
                    \includegraphics[width=\textwidth]{img/schemes_with_pins/NOR_or_w_pins.png}
                    \label{NOR:schemat_or_w_pins}
                \end{figure}
                \begin{figure}[H]
                    \centering
                    \includegraphics[width=\textwidth]{img/NOR/funkcje/1652306732459_scaled.png}
                    \caption{Zbudowany układ}
                    \label{NOR:zbudowany_układ_OR}
                \end{figure}
                
        \pagebreak            
                    
            \item Sprawdzanie stanów:
                \begin{figure}[H]
                    \centering
                        \begin{subfigure}[h]{0.4\textwidth}
                            \includegraphics[width=\textwidth]{img/NOR/funkcje/1652306732459_scaled.png}
                        \end{subfigure}
                        \begin{subfigure}[h]{0.4\textwidth}
                            \includegraphics[width=\textwidth]{img/NOR/funkcje/1652306732450_scaled.png}
                        \end{subfigure}
                        \begin{subfigure}[h]{0.4\textwidth}
                            \includegraphics[width=\textwidth]{img/NOR/funkcje/1652306732442_scaled.jpg}
                        \end{subfigure}
                        \begin{subfigure}[h]{0.4\textwidth}
                            \includegraphics[width=\textwidth]{img/NOR/funkcje/1652306732434_scaled.png}
                        \end{subfigure}
                    \label{NOR:testy_OR}
                \end{figure}
            \item Tabela stanów:
                \begin{center}
                    \label{NOR:tabela_prawdy_OR}
                    \begin{tabular}{|c|c|>{\columncolor[gray]{0.8}}c|}
                        \hline
                        A & B & Y \\
                        \hline
                        0 & 0 & 0 \\
                        \hline
                        0 & 1 & 1 \\
                        \hline
                        1 & 0 & 1 \\
                        \hline
                        1 & 1 & 1 \\
                        \hline
                    \end{tabular}
                \end{center}
            \item Zbudowany układ \textbf{poprawnie} reaguje na przesyłane sygnały (zgodnie z teoretycznymi przewidywaniami \ref{tabela_prawdy:OR})
        \end{itemize}
        
\pagebreak
        
    \section{AND}
            
        \begin{itemize}
            \item Zbudowany układ wraz ze schematem oraz rozpiską pinów:
                \begin{figure}[H]
                    \centering
                    \includegraphics[width=\textwidth]{img/schemes_with_pins/NOR_and_w_pins.png}
                    \label{NOR:schemat_and_w_pins}
                \end{figure}
                \begin{figure}[H]
                    \centering
                    \includegraphics[width=\textwidth]{img/NOR/funkcje/1652306732425_scaled.png}
                    \caption{Zbudowany układ}
                    \label{NOR:zbudowany_układ_AND}
                \end{figure}
                    
            \pagebreak        
                    
            \item Sprawdzanie stanów:
                \begin{figure}[H]
                    \centering
                        \begin{subfigure}[h]{0.4\textwidth}
                            \includegraphics[width=\textwidth]{img/NOR/funkcje/1652306732408_scaled.png}
                        \end{subfigure}
                        \begin{subfigure}[h]{0.4\textwidth}
                            \includegraphics[width=\textwidth]{img/NOR/funkcje/1652306732399_scaled.png}
                        \end{subfigure}
                        \begin{subfigure}[h]{0.4\textwidth}
                            \includegraphics[width=\textwidth]{img/NOR/funkcje/1652306732390_scaled.png}
                        \end{subfigure}
                        \begin{subfigure}[h]{0.4\textwidth}
                            \includegraphics[width=\textwidth]{img/NOR/funkcje/1652306732382_scaled.png}
                        \end{subfigure}
                    \label{NOR:testy_AND}
                \end{figure}
            \item Tabela stanów:
                \begin{center}
                    \label{NOR:tabela_prawdy_AND}
                    \begin{tabular}{|c|c|>{\columncolor[gray]{0.8}}c|}
                        \hline
                        A & B & Y \\
                        \hline
                        0 & 0 & 0 \\
                        \hline
                        0 & 1 & 0 \\
                        \hline
                        1 & 0 & 0 \\
                        \hline
                        1 & 1 & 1 \\
                        \hline
                    \end{tabular}
                \end{center}
            \item Zbudowany układ \textbf{poprawnie} reaguje na przesyłane sygnały (zgodnie z teoretycznymi przewidywaniami \ref{tabela_prawdy:AND})
        \end{itemize}


\chapter{Ćwiczenie 4}

\section{Wstęp do ćwiczenia}

\begin{itemize}
    \item Cwiczenie przeprowadzono na \textbf{Płytce RLC 15}
    \item Zbadano działanie dzielnika napięcia podając na wejście napięcia zmienne sinusoidalne z generatora przy ustalonej (\textbf{5kHz}) częstotliwości.
    \begin{figure}[h]
        \centering
        \includegraphics[scale=3]{images/dzielnik.png}
        \caption{Układ ilustrujący dzielnik}
        \label{fig:dzielnik}
    \end{figure}
    \item Zmierzono wartości rezystorów za pomocą \textbf{multimetra}.
    \begin{center}
        $R_1$ = \textbf{6.7k}\boldsymbol{\Omega} \\
        $R_2$ = \textbf{2.987k}\boldsymbol{\Omega}
    \end{center}
\end{itemize}

\section{Budowa dzielnika}

\begin{itemize}
    \item Zbudowany układ wygląda następująco:
    \begin{figure}[h]
        \centering
        \includegraphics[scale = 0.05]{images/IMG_20220316_113526.jpg}
        \includegraphics[scale = 0.05]{images/IMG_20220316_113538.jpg}
        \caption{Zbudowany dzielnik}
        \label{fig:my_label}
    \end{figure}
    
\end{itemize}

\section{Pomiary}
    
\begin{itemize}
    \item Teoretyczna wartość $U_{wy}$ wynosi:
    \begin{gather} \label{dzielnik_row}
        U_{wy} = U_{we} \cdot \frac{R_2}{R_1 + R_2} \\\\
        a_t = \frac{R_2}{R_1 + R_2} = \frac{2.987k\Omega}{9.687k\Omega} \approx \textbf{0.308}
    \end{gather}

    Napięcie wyjściowe jest zatem liniowo zależne od napięcia wejściowego.
\end{itemize}
\begin{center}
    \begin{tabular}{|c|c|c|}
         \hline
         $U_{we}$ [V] & Teoretyczna wartość $U_{wy}$ [V] & Zmierzona wartość $U_{wy}$ [V] \\
         \hline
         1 & 0.308 & 0.296\\
         \hline
         2 & 0.616 & 0.616\\
         \hline
         3 & 0.924 & 0.92 \\
         \hline
         4 & 1.232 & 1.22 \\
         \hline
         5 & 1.54  & 1.48 \\
         \hline
         6 & 1.848 & 1.8  \\
         \hline
         7 & 2.156 & 2.08 \\
         \hline
         8 & 2.464 & 2.46 \\
         \hline
         9 & 2.772 & 2.68 \\
         \hline
         10 & 3.08 & 2.96 \\
         \hline
    \end{tabular}
\end{center}



\begin{center}
    \begin{tikzpicture}
        \begin{axis}[
            axis lines = left,
            xlabel = \(U_{we}\),
            x unit = V,
            xtick = {0,1,2,3,4,5,6,7,8,9,10},
            ylabel = {\(U_{wy}\)},
            y unit = V,
            xmin=0, xmax=10,
            ymin=0, ymax=4,
        ]
        \addplot [
            domain=0:10, 
            samples=100, 
            color=red,
        ]
        {0.308 * x};
        \addlegendentry{Teoretyczne wartości}

        \addplot [
            only marks,
            color=blue,
            mark=*,
            ]
            coordinates {
                (1, 0.296)(2,0.616)(3,0.92)(4,1.22)(5,1.48)(6,1.8)(7,2.08)(8,2.46)(9, 2.68)(10, 2.96)
            };
        \addlegendentry{Zmierzone wartości}
        \end{axis}
    \end{tikzpicture}
\end{center}

\pagebreak

\begin{center}
    \textbf{Pomiary dla $U_{we}$ = 1V, 5V, 10V}
\end{center}
\begin{figure}[h]
    \centering
    \includegraphics[scale=0.34]{images/1_7-dzielnik_1V.png}
    \includegraphics[scale=0.34]{images/1_7-dzielnik_5V.png}
    \includegraphics[scale=0.34]{images/1_7-dzielnik_10V.png}
    \label{fig:pomiary_dzielnika}
\end{figure}

\section{Podsumowanie}

\begin{itemize}
    \item Biorąc pod uwagę zmierzone wartości:
    \begin{gather}
       a_z = \frac {\sum^{n} \frac{U_{wy}}{U_{we}}}{n} \\\\
       a_z = \frac{3.01}{10} \\\\
       a_z = \textbf{0.301}
    \end{gather}
    Średni błąd wyniósł:
    \begin{center}
        $|a_t - a_z|$ = \textbf{0.007} \\
    \end{center}
    
    \item Uzyskane pomiary nie różniły się w dużym stopniu od teoretycznych wartości
\end{itemize}



\end{document}