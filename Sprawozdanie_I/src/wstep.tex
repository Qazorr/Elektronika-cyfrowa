\chapter{Wstęp}

\section{Używany sprzęt}
\begin{itemize}
    \item Stanowisko nr 7
    \item Oscyloskop - MSO3012 
    \item Generator funkcyjny - AFG3022B
    \item Płytka RLC - numer 15
\end{itemize}

\section {Jednostki i przedrostki}

\begin{itemize}
\item 1 \boldsymbol{\Omega} (om) - wielkość rezystancji - 1\boldsymbol{\Omega} = {$\frac{1V}{1A}$} ($\frac{wolt}{amper}$)
\item 1 V (wolt) - jednostka potencjału elektrycznego, napięcia elektrycznego i siły elektromotorycznej - 1V = $\frac{1W}{1A}$ ($\frac{wat}{amper}$)
\item 1 A (amper) - jednostka natężenia prądu elektrycznego - 1A = $\frac{1C}{1s}$
\item 1 Hz (herc) - jednostka miary częstotliwości - 1Hz = $\frac{1}{1s} = 1s^{-1}$
\end{itemize}

\begin{itemize}
    \item k (kilo) = $10^3$
    \item m (mili) = $10^{-3}$
\end{itemize}


\section {Oscyloskop}
Przyrząd elektroniczny służący do obserwowania, obrazowania i badania przebiegów
zależności pomiędzy dwiema wielkościami elektrycznymi, bądź innymi wielkościami fizycznymi
reprezentowanymi w postaci elektrycznej.
\section {Generator funkcyjny}

Urządzenie generujące oscylujące napięcie przy zadanych parametrach
częstotliwości, amplitudy oraz kształtu sygnału.

\section{Krzywe Lissajous}

Krzywa parametryczna, która jest wykreślana przez punkt materialny wykonujący
drgania harmoniczne w dwóch wzajemnie prostopadłych kierunkach dana wzorem:
\begin{equation}
    \begin{cases}
        x(t) = Asin(at + \delta) \\
        y(t) = Bsin(bt)
    \end{cases}
\end{equation}

\section {Multimetr (Miernik uniwersalny)}

Zespolone urządzenie pomiarowe posiadające możliwość pomiaru różnych wielkości fizycznych

\section {Zjawisko dudnień}

Zjawisko powstające w wyniku nałożenia się dwóch drgań o takich samych amplitudach i nieznacznie różniących się częstotliwościach.

\section {Łączenie rezystorów}

Szeregowe:
\begin{equation} \label{szereg}
    R_z = \sum^{n}_{i=1}{R_i}
\end{equation}
Równoległe:
\begin{equation}
    \frac{1}{R_z} = \sum^{n}_{i=1}{\frac{1}{{R_i}}}
\end{equation}

\section {Prawo Ohma}

Prawo fizyki głoszące proporcjonalność natężenia prądu płynącego przez przewodnik do napięcia panującego między końcami przewodnika.\\
Dla prądu stałego proporcjonalność napięcia U i natężenia I wyraża się wzorem:
\begin{equation}
    U = I * R
\end{equation}

\section {Dzielnik napięcia}

Obwód dostarczający na wyjściu napięcie będące ułamkiem napięcia wejsciowego. 