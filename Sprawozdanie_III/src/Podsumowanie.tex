\chapter{Podsumowanie}

\section{Opis doświadczeń}
\begin{itemize}
    \item Celem przeprowadzonych doświadczeń było zapoznanie się ze wzmacniaczem operacyjnym, znajdującym się na płytce.
    \item Zbudowano 4 układy: wzmacniacz odwracający fazę, sumator napięć, przerzutnik Schmidta oraz multiwibrator astabilny.
    \item Wszystkie układy działały poprawnie. 
    \item Wszelakie pomiary zgadzały się z teoretycznymi przewidywaniami, mimo odchyleń prawdopodobnie spowodowanych złą kalibracją sprzętu, bądź zbyt krótkim czasem poszczególnych pomiarów, w których oscyloskop nie zdołał poprawnie uśrednić wszystkich pomiarów.
    \item Przez brak czasu nie udało się przeprowadzić w pełni doświadczeń na multiwibratorze astabilnym.
\end{itemize}

\section{Literatura}

\begin{enumerate}
    \item IS Ćwiczenie 3 slajdy - dr Szymon Niedźwiecki
    \item Studencka Pracownia Elektroniczna \\
    Wykład "Wstęp do elektroniki" dr hab. Janusza Brzychczyka \\
    \url{http://zefir.if.uj.edu.pl/pracownia_el/jb_w6.pdf}
\end{enumerate}