\chapter{Podsumowanie}

\section{Opis doświadczeń}

\begin{itemize}
    \item Płytka \textbf{poprawnie} reaguje na podawane impulsy.
    \item Układy TTL (7400, 7402, 7486) \textbf{poprawnie} reagowały na podawane impulsy, tabele wyjść zgadzały się z teoretycznymi.
    \item Zbudowane funkcje logiczne (NOT, OR, AND) za pomocą funktorów NAND oraz NOR generowały \textbf{poprawne} tablice prawd.
    \item Czas propagacji \textbf{zgadzał się} z przewidywanymi wartościami.
    \item Zbudowany asynchroniczny przerzutnik R-S \textbf{zachowywał się zgodnie z przewidywaniami}. Po odłączeniu sygnałów wejściowych przerzutnik pamiętał ostatnio zapisany stan.
    \item Zbudowany na laboratoriach układ realizujący \textbf{funkcję logiczną segmentu c} generował \textbf{niepoprawną} tablicę prawd.
\end{itemize}

\section{Linki}
\begin{itemize}
    \item Dokumentacja TTL \textbf{7400}: \\ 
        \label{dokumentacja:7400} \url{http://zefir.if.uj.edu.pl/pracownia_el/7400.pdf}
    \item Dokumentacja TTL \textbf{74S00}: \\
        \label{dokumentacja:74S00} \url{http://www.datasheet39.com/PDF/684118/74S00-datasheet.html}
    \item Dokumentacja TTL \textbf{7402}: \\
        \label{dokumentacja:7402} \url{http://zefir.if.uj.edu.pl/pracownia_el/7402.pdf}
    \item Dokumentacja TTL \textbf{MC74HC86}: \\
        \label{dokumentacja:mc74hc86} \url{https://www.onsemi.com/pdf/datasheet/mc74hc86a-d.pdf}
    \item Studencka Pracownia Elektroniczna \\
        Wykład "Wstęp do elektroniki" dr. hab. Janusza Brzychczyka \\ 
        \url{https://spe.if.uj.edu.pl/documents/144978191/147822261/T_07a.pdf/0e35aa24-8efa-4ad7-9f76-c9665291215d}
\end{itemize}